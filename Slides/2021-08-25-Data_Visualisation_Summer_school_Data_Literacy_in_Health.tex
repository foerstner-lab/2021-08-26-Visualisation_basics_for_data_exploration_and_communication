\documentclass[aspectratio=169]{beamer}
\usepackage[utf8]{inputenc}
\usepackage{tikz} % For semi-transparent background blocks
\usepackage{hyperref}
\usepackage{multimedia}
\usepackage{ulem}
\usepackage{wasysym}

\usepackage{pbox}

\usepackage[absolute,overlay]{textpos}

\usepackage{smartdiagram}
\usetikzlibrary{shapes.geometric,calc}
\usetikzlibrary{shapes.symbols}
\usetikzlibrary{shapes.symbols,positioning}
\usepackage{metalogo}

\usetikzlibrary{backgrounds, calc, shadows, shadows.blur}

\newcommand\addcurlyshadow[2][]{
    % #1: Optional aditional tikz options
    % #2: Name of the node to "decorate"
    \begin{pgfonlayer}{background}
        \path[blur shadow={shadow xshift=0pt, shadow yshift=0pt, shadow blur steps=6}, #1]
        ($(#2.north west)+(.3ex,-.5ex)$)
        -- ($(#2.south west)+(.5ex,-.7ex)$)
        .. controls ($(#2.south)!.3!(#2.south west)$) .. (#2.south)
        .. controls ($(#2.south)!.3!(#2.south east)$) .. ($(#2.south east)+(-.5ex,-.7ex)$)
        -- ($(#2.north east)+(-.3ex, -.5ex)$)
        -- cycle;
    \end{pgfonlayer}
}

%%%%%%%%%%%%%%%%%%%%%%%%%%%%%%%%%%%%%%%%%%%%%%%%%%%%%%%%%%%%%%%%%%%
% Style modifications
%%%%%%%%%%%%%%%%%%%%%%%%%%%%%%%%%%%%%%%%%%%%%%%%%%%%%%%%%%%%%%%%%%%

\usetheme{Berlin}

%%% Fonts %%%

% Change font. Fontspec requires xelatex instead of pdflatex!
% Font catalog: http://www.tug.dk/FontCatalogue/
\usepackage{fontspec}

%\setsansfont{Comfortaa}
%\setsansfont{DejaVu Sans}
%\setsansfont{Fira Sans}

% Use "Fira Sans Light" as the normal font and the "Fira Sans" for
% bold fonts

\setsansfont[
  ItalicFont={Fira Sans Light Italic},
  BoldFont={Fira Sans},
  BoldItalicFont={Fira Sans Italic}]{Fira Sans Light}

\setbeamerfont{title}{size=\Large, series=\bfseries}
\setbeamerfont{frametitle}{size=\large, series=\bfseries}

%%% Slide template %%%

\setbeamertemplate{frames}[default]

% Empty headline / footline
\setbeamertemplate{headline}{}
\setbeamertemplate{footline}{}

% Remove navigation icons
\setbeamertemplate{navigation symbols}{}

%%% Colors %%%

\usecolortheme{crane}

\definecolor{lightgray}{RGB}{220,220,220}
\definecolor{darkgray}{RGB}{45,45,45}
%\definecolor{darkblue}{RGB}{0,86,137}
%\definecolor{darkblue}{RGB}{22,90,151}
\definecolor{lightblue}{RGB}{229, 245, 255}

%\definecolor{darkblue}{RGB}{1,1,100}


% Use the slide background in block environments                                                                                                      
\setbeamercolor{title}{fg=white,bg=darkgray}
\setbeamertemplate{blocks}[default]
\setbeamercolor{block title}{bg=}
\setbeamercolor{block body}{bg=lightgray}
\setbeamercolor{frametitle}{fg=white,bg=darkgray}
\setbeamerfont{block body}{size=\large}
\setbeamercolor{itemize item}{fg=black}
\setbeamertemplate{itemize items}[circle]
\setbeamercolor{section number projected}{bg=darkgray,fg=white}
\setbeamercolor{section in toc}{fg=black}
\setbeamercolor{subsection in toc}{fg=darkgray}

\addtobeamertemplate{block begin}{\pgfsetfillopacity{0.8}}{\pgfsetfillopacity{1}}
\addtobeamertemplate{frametitle}{\pgfsetfillopacity{1.0}}{\pgfsetfillopacity{1}}
\addtobeamertemplate{title page}{\pgfsetfillopacity{1.0}}{\pgfsetfillopacity{1}}

%%% Misc %%%

% Command to place the test (e.g. citation) in the center of the footer
\newcommand{\setfootercentertext}[1]{
\setbeamertemplate{footline}{
  \hspace*{\fill}
  \raisebox{3mm}[0mm][0mm]{
    \tiny{#1}}\hspace*{\fill}}
}

%%%%%%%%%%%%%%%%%%%%%%%%%%%%%%%%%%%%%%%%%%%%%%%%%%%%%%%%%%%%%%%%%%%
% Content
%%%%%%%%%%%%%%%%%%%%%%%%%%%%%%%%%%%%%%%%%%%%%%%%%%%%%%%%%%%%%%%%%%%

%------------------------------------------------------------------------------
\title{Visualization basics for \\data exploration and communication}
%------------------------------------------------------------------------------

\author{\small Prof. Dr. Konrad U. Förstner}

%\institute{ZB MED -- Information Center Life Sciences, Cologne, Germany \&\\
%  TH Köln -- University of Applied Sciences, Cologne Germany}

\date{\scriptsize
  Virtual Summer School \textit{Data Literacy in Health}, August 26 2021\\ \ \\
  % \includegraphics[width=1.0cm]{images/vcard_qr_code.png}\ \\ \ \\
  \includegraphics[width=1.0cm]{images/creative_commons_attribute.png}
}

\logo{
  \includegraphics[height=1.0cm]{images/ZBMED_2017_EN.pdf}
  \includegraphics[height=0.8cm]{images/logo_TH-Koeln_CMYK_22pt.eps}
}


\begin{document}


\begin{frame}{}
  \titlepage
\end{frame}
\logo{}


\setbeamertemplate{background}{}
\setbeamertemplate{footline}{}

\setbeamertemplate{background}{}
\setfootercentertext{}
\begin{frame}
  \frametitle{}
  \begin{center}
    Slides and Code (Jupyter Notebooks) can be found here:\\
    \ \\
    \includegraphics[width=3cm]{images/qr_to_git_repo.eps}\\
    \ \\
    \href{https://github.com/foerstner-lab/2021-08-26-Visualisation_basics_for_data_exploration_and_communication}{https://github.com/foerstner-lab/2021-08-26-Visualisation\_basics\_for\_data\_exploration\_and\_communication}\\
    \ \\
    \href{https://bit.ly/2XODDoM}{https://bit.ly/2XODDoM}
  \end{center}

\end{frame}

\setbeamertemplate{background}{}
\setfootercentertext{}

\begin{frame}
  \frametitle{Plan for this session}
  \begin{center}
    {\LARGE
    \begin{tabular}{lll}
      09:00 - 10:00 & Introduction \& Live Coding\\
      10:00 - 11:00 & Working on your data\\
      11:00 - 12:00 & Discussion, Questions \& Answers\\
    \end{tabular}
    }
  \end{center}
\end{frame}


\begin{frame}
  % \frametitle{Basic concept of supervised machine learning}
  \begin{block}{}
    \vspace{0.5cm}
    \ \ \ \
    \begin{minipage}{0.10\textwidth}
      \begin{center}
        \includegraphics[width=1.6cm]{images/publicdomainvectors_target-plain.pdf}
      \end{center}        
    \end{minipage}
    \hfill
    \begin{minipage}{0.80\textwidth}
      After this session you should have a basic understanding of
      selected concepts of data visualization that are independent
      of the tool that you use.
    \end{minipage}
    \vspace{0.3cm}
  \end{block}
\end{frame}

%%%%%%%%%%%%%%%%%%%%%%
\section{Introduction \& Live Coding}
%%%%%%%%%%%%%%%%%%%%%%

\begin{frame}{}
 \tableofcontents
\end{frame}

\begin{frame}{}
  \tableofcontents[currentsection]
\end{frame}

\setbeamertemplate{background}{
\includegraphics[width=\paperwidth]{images/flickr_capcase_4970062156_Swiss_Army_Knife.jpg}}
\setbeamertemplate{footline}{\raisebox{2mm}[2mm][2mm]{\Tiny{\
       \href{https://www.flickr.com/photos/capcase/4970062156/}{https://www.flickr.com/photos/capcase/4970062156/}
      -- CC-BY by flickr user \href{http://www.flickr.com/photos/capcase/}{capcase}}}}
\begin{frame}
  % \frametitle{A swiss army knife for systems biology}
  \begin{block}{}
    \begin{center}
      Visualisation are powerful tools to \\understand data and
      communicate ideas.
    \end{center}
  \end{block}
\end{frame}
\setbeamertemplate{background}{}
\setbeamertemplate{footline}{}

\setbeamertemplate{background}{}
\setfootercentertext{}
\begin{frame}
  \frametitle{Visualising ideas, plans or concepts}
  \begin{center}
    \includegraphics[width=8.5cm]{images/sRNARegNet_Working_plan.pdf}\\
  \end{center}
\end{frame}


\setfootercentertext{
  Berger, \textit{et al.}, 2016, \textit{Sci. Rep.},
  \href{https://doi.org/10.1038/srep35307}{https://doi.org/10.1038/srep35307}
}
\begin{frame}{}
  \frametitle{Visualising research results}
  \begin{center}
    \includegraphics[height=5.5cm]{images/Berger_Knoedler_Foerstner_et_al_Fig_1.jpg}\\
  \end{center}  
\end{frame}

\setbeamertemplate{background}{}
\setfootercentertext{}
\begin{frame}
  %\frametitle{Visualising ideas, plans or concepts}
  \begin{center}
    \includegraphics[width=12cm]{images/Data_encode_visualisation_decode.pdf}\\
  \end{center}
\end{frame}

\begin{frame}
  \frametitle{Entities and their features}    
  \begin{block}{}
    \vspace{0.5cm}
    \ \ \ \
    \begin{minipage}{0.10\textwidth}
      \begin{center}
        \includegraphics[width=1.6cm]{images/publicdomainvectors_ftdissociatecell.pdf}
      \end{center}        
    \end{minipage}
    \hfill
    \begin{minipage}{0.80\textwidth}

      Features can be\\
      \begin{itemize}
        \item categorical / qualitative
          \begin{itemize}
          \item Nominal (e.g. cell line, cancer type, eye color, gender)
          \item Ordinal (e.g. very bad, bad, good, very good)
          \end{itemize}
        \item numerical / quantitative
          \begin{itemize}
          \item Discrete (e.g. gene length in nucleotides, number cells)
          \item Continuous (e.g. cell length, concentration, relative expression) 
          \end{itemize}
      \end{itemize}
      
    \end{minipage}
    \vspace{0.3cm}
  \end{block}
\end{frame}


\setbeamertemplate{background}{
  \includegraphics[width=\paperwidth]{images/Tools_by_Todd_Quackenbush.jpg}}
\setfootercentertext{
      \href{https://unsplash.com/@toddquackenbush?photo=IClZBVw5W5A}{
        https://unsplash.com/@toddquackenbush?photo=IClZBVw5W5A} - PD}
\begin{frame}
\end{frame}

\setbeamertemplate{background}{}
\setfootercentertext{}

\setfootercentertext{
  Mackinlay,  1986, \textit{ACM Transactions on Graphics},
  \href{https://doi.org/10.1145/22949.22950}{https://doi.org/10.1145/22949.22950}
}

\begin{frame}
  \frametitle{}
  \begin{center}
    \includegraphics[width=5.0cm]{images/visualisation_accuracy.jpg}\\
  \end{center}
\end{frame}


\setbeamertemplate{background}{}
\setfootercentertext{}

\begin{frame}
  \frametitle{}
  \begin{center}
    {\Huge Live Coding }
  \end{center}
\end{frame}



\logo{}
\setbeamercolor{block body}{bg=lightgray}
\setbeamertemplate{background}{
  \includegraphics[width=\paperwidth]{images/flickr_nateone_3768979925.jpg}}
\setfootercentertext{
  \href{https://www.flickr.com/photos/nateone/3768979925/}{https://www.flickr.com/photos/nateone/3768979925/}
  -- CC-BY by flick user
  \href{https://www.flickr.com/photos/nateone/}{nateone}}
\begin{frame}
  % \frametitle{Acknowledgements}
  \begin{block}{}
    \begin{center}
      \textbf{What are your questions?}\\
      \ \\
      \href{https://zbmed.de}{zbmed.de} / \href{https://twitter.com/ZB_MED}{@ZB\_MED} \\      
      \ \\
      \href{https://twitter.com/konradfoerstner}{@konradfoerstner}\\
      \ \\
      \includegraphics[height=1.8cm]{images/ZBMED_2017_EN.pdf} 
      \includegraphics[height=0.8cm]{images/logo_TH-Koeln_CMYK_22pt.eps}
      \\
    \end{center}
  \end{block}
\end{frame}

%%%%%%%%%%%%%%%%%%%%%%%%%%%%%
\section{Working on your data}
%%%%%%%%%%%%%%%%%%%%%%%%%%%%%

\setbeamertemplate{background}{}
\setfootercentertext{}

\begin{frame}{}
  \tableofcontents[currentsection]
\end{frame}

\begin{frame}
  \frametitle{Visualize your data}
  \begin{block}{}  
    \begin{center}
      \begin{itemize}
      \item Which types of feature are in you data set?
      \item Which features would you like to visualize?
      \item How do you want to encode features and combine their
        visualization?
      \end{itemize}
    \end{center}
  \end{block}    
\end{frame}



%%%%%%%%%%%%%%%%%%%%%%%%%%%%%
\section{Discussion, Questions \& Answers}
%%%%%%%%%%%%%%%%%%%%%%%%%%%%%

\begin{frame}{}
  \tableofcontents[currentsection]
\end{frame}

%%%%%%%%%%%%%%%%%%%%%%%%%%%%%
\section{Reading recommendations}
%%%%%%%%%%%%%%%%%%%%%%%%%%%%%


\begin{frame}
  \frametitle{Reading recommendations}
  \begin{block}{}  
    \begin{center}
      \begin{itemize}
      \item The
        \href{http://blogs.nature.com/methagora/2013/07/data-visualization-points-of-view.html}{\textit{Points
            of View}} Series in \textit{Nature Methods}
      \item \textit{Fundamentals of Data Visualization}, Claus Wilke,
        2019, O’Reilly Media, ISBN-13: 978-1492031086
      \item \textit{Visualization Analysis and Design}, Tamara
        Munzner, 2014, Taylor \& Francis, ISBN-13: 978-1466508910
      \end{itemize}
    \end{center}
  \end{block}    
\end{frame}

\end{document}
